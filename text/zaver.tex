\chapter{Závěr}
\addcontentsline{toc}{chapter}{Závěr}

Podařilo se nám implementovat webovou aplikaci pro~vyhledávání ilustračních obrázků z~textu. Obrázky pochází z~datasetu Profimedie. Každý obrázek má přiřazena klíčová slova, která aplikace používá k~vyhledávání. Podporuje český a anglický jazyk, ale je snadno rozšiřitelná na podporu jiných jazyků. Uživatel může nejen zadat text článku, ale může si i vynutit přímo klíčová slova u hledaných obrázků.

Klíčovým algoritmem pro~vyhledávání je algoritmus extrakce klíčových slov. Ten je založen na modifikované metodě TF-IDF. Extrahovaná klíčová slova se používají k~vyhledávání doporučených obrázků. Druhé využití vyextrahovaných klíčových slov z~textu je jako nápověda uživateli.

Zdrojová data v~datasetu Profimedia jsou v~angličtině. Aby aplikace byla schopná pracovat s českými texty, bylo nutné klíčová slova obrázků přeložit. Použili jsme hybridní metodu, která kromě slovníkového překladu využívá i překlad detekovaných víceslovných frází v~textu. Menší kvalita takového překladu způsobuje občasné problémy algoritmu při doporučování obrázků k~českým textům. Při tak velkém množství dat by ovšem kvalitnější překlad byl velmi náročný~na zdroje, nebo příliš drahý.

Součástí práce je i rozbor některých moderních možností tvorby frontendu a backendu webových aplikací. Práce vysvětluje některé novinky v~HTML5 a ukazuje využití javascriptových frameworků pro~různé účely. Na backendové straně aplikace ukazuje některé nové komunikační a datové protokoly. Rozebrány jsou některé možnosti uložení velkého množství prohledavatelných dat.

Kromě textového vyhledávání umožňuje aplikace i vyhledávání podle vizuální podobnosti obrázků. Uživatel aplikace vidí v~detailu každého z~nalezených obrázků množinu podobných obrázků. v~rámci aplikace byla vytvořena nezávislá služba, která k~obrázkům v~datasetu Profimedia hledá obrázky. Podobnost obrázků je určena vzdáleností vektorů se $4\ 096$ složkami. Jelikož obrázků v~datasetu Profimedia je více než 20 milionů, bylo největším úkolem implementovat službu doporučování obrázků tak, aby vracela kvalitní výsledky co nejrychleji.

Algoritmus doporučování ilustračních obrázků byl testován s lidskými anotátory. Anotátoři dostávali české texty z~online médií a množinu obrázků. Některé obrázky byly získané algoritmem na doporučení obrázků k~textu. Jednu anotaci dostali nezávisle na sobě vždy dva anotátoři. v~prvním úkolu dostali anotátoři jeden náhodný a čtyři doporučené obrázky. z~58 anotací označili anotátoři správně náhodně vybraný výsledek ve~45 případech. Ukázalo se ovšem, že metoda testování má některé problémy, které umožnily anotátorům vybrat náhodný obrázek aniž by ostatní obrázky byly k~textu relevantní. Druhá metoda testování otočila poměr a anotace obsahovaly čtyři náhodné a jeden doporučený obrázek. Úkolem anotátorů bylo označit právě jeden nenáhodný obrázek. Každou anotaci opět dostali nezávisle na sobě vždy dva anotátoři. Ze 120 anotací se oba anotátoři shodli na správném obrázku v~76 případech. Rozbor anotací u kterých se oba anotátoři neshodli ukazuje různé příčiny. Některá klíčová slova k~obrázkům jsou chybně přeložena. Některé obrázky obsahují špatná klíčová slova. Klíčová slova některých obrázků jsou příliš krátká. v~několika anotacích by některé z~náhodně vybraných obrázků mohly být ilustrační obrázky k~textu také, což negativně ovlivnilo výsledky testování v~neprospěch algoritmu. pro~anglické články testování neproběhlo, ale je pravděpodobné, že dosažené výsledky by byly lepší --- u anglických klíčových slov nebylo potřeba použít automatický překlad.

Testování ukázalo, že aplikace ve~vysokém procentu nabízí uživatelům přijatelné ilustrační obrázky. Pokud jsou výsledky nevyhovující, může uživatel využít nápovědu doporučených klíčových slov a hledaný výsledek snadno přesněji specifikovat. ve~výsledku by tedy aplikace měla být schopna usnadnit proces vyhledávání ilustračních obrázků ke zpravodajským článkům, což bylo hlavním cílem práce.

