%%% Seznam použité literatury je zpracován podle platných standardů. Povinnou citační
%%% normou pro diplomovou práci je ISO 690. Jména časopisů lze uvádět zkráceně, ale jen
%%% v kodifikované podobě. Všechny použité zdroje a prameny musí být řádně citovány.

\def\bibname{Seznam použité literatury}


\begin{thebibliography}{99}
\addcontentsline{toc}{chapter}{\bibname}

\bibitem{botorek}
  BOTOREK, Jan. \textit{Tvorba nástroje pro zpracování textových popisů multimediálních dat} [online]. 2012 [cit. 2014-07-17]. Bakalářská práce. Masarykova univerzita, Fakulta informatiky. Vedoucí práce Petra Budíková. Dostupné z: \url{http://is.muni.cz/th/359815/fi_b/}.

\bibitem{brno}
  BUDÍKOVÁ, Petra, Michal BATKO a Pavel ZEZULA. \textit{Evaluation Platform for Content-based Image Retrieval Systems}. In \textit{International Conference on Theory and Practice of Digital Libraries 2011, LNCS 6966}. Berlin: Springer, 2011. s.~130-142, 12 s. ISBN~978-3-642-24468-1.

\bibitem{chang}
  CHANG, Jonathan, et al. \textit{Reading tea leaves: How humans interpret topic models}. In: \textit{Advances in neural information processing systems}. 2009. s. 288-296.

\bibitem{caffee}
  JIA, Yangqing. \textit{Caffe: An open source convolutional architecture for fast feature embedding}. 2013
  Dostupné z: \url{http://caffe.berkeleyvision.org}.

\bibitem{lott}
  LOTT, Brian. \textit{Survey of Keyword Extraction Techniques}. UNM Education, 2012.

\bibitem{wordnet}
  MILLER, George A.
  \emph{WordNet: A Lexical Database for English}.
  Communications of the ACM, 1995, roč. 38, č. 11, s. 39-41.


\bibitem{porter80}
  PORTER, Martin F.
  \emph{An algorithm for suffix stripping}.
  Program: electronic library and information systems. MCB UP Ltd, 1980, roč. 14, č. 3, s. 130-137.

\bibitem{geohash}
  Wikipedia
  \emph{Geohash --- Wikipedia{,} The Free Encyclopedia} [online]. 2014 [cit. 2014-07-24]
  Dostupné z: \url{http://en.wikipedia.org/w/index.php?title=Geohash&oldid=609521802}.


\end{thebibliography}

%\inputencoding{utf8}


