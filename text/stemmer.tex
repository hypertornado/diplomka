\chapter{Stemmer}

Důležitou vlastností systémů na vyhledávání v textu je, aby dokázali najít i v jiných tvarech slova. Pokud uživatel hledá slovo \uv{praha}, většinou očekává, že se mu zobrazí i výsledky obsahující slovo \uv{praze}. Je tedy potřeba mít nějaký algoritmus, který k sobě slova jako \uv{praha} a \uv{praze}.

První možností je použít lemmatizér. Úkolem lemmatizéru je ke každému slovu přiřadit jeho základní tvar. U podstatných jmen je to většinou první pád jednotného čísla (\uv{praha}), u sloves infinitiv.

Alternativou lemmatizátoru může být stemmer. Ten většinou používá jednoduché heuristiky k odstranění koncovek slov. Stemmer nemusí vrátit regulérní slovo jazyka, například pro slova \uv{praha} a \uv{praze} může stemmer vrátit slovo \uv{prah}. Výhodou stemmeru oproti lemmatizéru je, že většinou používá pouze jednoduché heuristiky. Je tedy většinou rychlejší a méně náročný na zdroje než lemmatizér. Ve vyhledávacích aplikacích je stemmer dostačující.

Pro angličtinu je nejznámějším stemmerem pro angličtinu je Porterův stemmer popsaný\cite{porter1980algorithm} Martinem Porterem již v roce 1980. Kromě oficiální implementace existují porty do různých jazyků včetně Ruby. Tato aplikace využívá implementaci z Ruby Gemu \uv{stemmify}\footnote{\url{https://rubygems.org/gems/stemmify}} v licenci MIT.


\section{Český stemmer}
Pro jazyky jako je čeština, která má bohatší morfologii než angličtina, je tvorba stemmeru náročnější. Testovali jsme několik implementací českých stemmerů. Jako nejkvalitnější byla nakonec vybrána implementace českého stemmeru v knihovně Lucene. Tuto implementaci využívá i Elasticsearch.

V rámci této práce byl portován soubor \lstinline{CzechStemmer.java} ze zdrojového kódu knihovny Lucene do jazyka Ruby. Výsledkem je Ruby Gem pod licencí MIT, který lze použít nezávisle na zbytku aplikace. Instaluje se příkazem

\begin{lstlisting}
gem install czech-stemmer
\end{lstlisting}

Knihovna obsahuje pouze jednu třídu CzechStemmer s funkcí stem, která přijímá i vrací řetězec:

\begin{lstlisting}[language=ruby]
require 'czech-stemmer'

CzechStemmer.stem("praha") # => "prah"
CzechStemmer.stem("praze") # => "prah"
CzechStemmer.stem("předseda") # => "předsd"
CzechStemmer.stem("mladými") # => "mlad"
\end{lstlisting}
