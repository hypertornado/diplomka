\chapter*{Úvod (zatím opsané zadání)}
\addcontentsline{toc}{chapter}{Úvod}

Většina zpravodajských serverů často opatřuje publikované články tzv. ilustračními snímky, jejichž úkolem je vizuálně dokreslovat obsah článku a upoutat na něj čtenářovu pozornost. Ilustrační snímky většinou pocházejí z rozsáhlých fotografických databází, jsou vybírány autory článku a s obsahem článku souvisejí jen relativně volně. Výběr ilustračních snímků probíhá nejčastěji na základě porovnávání klíčových slov specifikovaných autorem textu a popisků, kterými jsou obrázky v databázi opatřeny (typicky svými autory). 

Proces výběru ilustračních snímků (dotazování ve fotografické databázi) je obtížný jednak pro samotný vyhledávací systém (hledání relevantních fotografií na základě uživatelských dotazů), jednak pro autory, kteří musí dotazy vytvářet. Konstrukce dotazů spočívá v několika krocích: uživatel nejdříve musí identifikovat ústřední téma (či témata) článku, které chce ilustrovat vhodnou fotografií, a ta potom popsat vhodnými klíčovými slovy, zvolit a zkombinovat je tak, aby vedla k nalezení vhodného obrázku. Tento proces by mohl být zjednodušen tím, že konstrukce dotazů pro vyhledávání bude prováděna automaticky pouze na základě textu článku. 

Cílem diplomové práce je navržení a implementace komfortní webové aplikace pro automatické navrhování ilustračních snímků na základě textu článku, bez nutnosti explicitně konstruovat vyhledávací dotazy. Součástí práce bude i uživatelská evaluace celého systému. Pro experimenty bude použita kolekce ilustračních snímků od společnosti Profimedia.

