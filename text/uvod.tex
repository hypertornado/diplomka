\chapter{Úvod}

Cílem diplomové práce je implementovat webovou aplikaci pro~doporučování ilustračních obrázků k~danému textu. Název aplikace je je \uv{Stock Photo Finder}. V~této práci ji budeme dále označovat slovem \uv{aplikace}. Vytvořit takovou aplikaci přináší mnoho rozličných úkolů a problémů.

\section{Práce s daty}
Vstupní data z~datasetu Profimedie obsahují více než 20 milionů anotací obrázků. Základním úkolem je být schopen takové množství dat nahrát do~databáze a v~tak velkých datech vyhledávat. Kapitola~\ref{chap:backend} rozebírá některé moderní programovací jazyky a databáze a jejich vhodnost pro~práci s velkými daty.

\section{Vyhledávací algoritmy}

Klíčovou součástí aplikace je algoritmus pro~extrakci klíčových slov a algoritmus pro~vyhledávání v~datasetu Profimedie. Tyto algoritmy jsou popsané v~Kapitole~\ref{chap:teorie}. Praktická implementace algoritmů a proces získávání potřebných dat je popsán v~Kapitole~\ref{chap:implementace}.


\section{Překlad}

Popisky klíčových slov jsou v~datasetu Profimedie v~anglickém jazyce. Cílem aplikace je umožnit vyhledávání doporučených ilustračních obrázků k~textu i pro~texty v~jiných jazycích, než je angličtina. Kapitola~\ref{chap:preklad} rozebírá překladové problémy z~teoretického i praktického hlediska. Sekce~\ref{subsec:zprovozneni_preklad} ukazuje praktickou při překladu do~češtiny. Sekce~\ref{subsec:zprovozneni_podpora} ukazuje postup pro~zprovoznění aplikace i v~dalších jazycích.


\section{Webová aplikace}

Výsledná aplikace je webovou aplikací. Kapitola~\ref{chap:backend} popisuje problémy při vývoji backendové části. Kapitola~\ref{chap:frontend} popisuje moderní technologie pro~vývoj frontendové části webových aplikací.


\section{Vyhledávání podobných obrázků}

Aplikace umožňuje kromě textového vyhledávání hledat ilustrační obrázky i podle vizuální podobnosti. Kapitola~\ref{chap:similar} popisuje algoritmy, které toto vyhledávání umožňují.

\section{Testování}

Výsledky detekce algoritmu byly testovány na uživatelích. pro~testování byla vyvinuta speciální webová aplikace popsaná v~Kapitole~\ref{chap:rozhrani}. Samotné testování algoritmu a jeho výsledky jsou popsány v~Kapitole~\ref{chap:evaluace}.


\section{Ostatní problémy}

Při práci na aplikaci bylo řešeno množství drobnějších problémů. Kapitola~\ref{chap:stemmer} popisuje stemming a vlastní implementaci stemmeru pro~český jazyk. Kapitola~\ref{chap:detekce} popisuje algoritmus pro~automatickou detekci jazyka textu. Kapitola~\ref{chap:rozhrani} popisuje anotační rozhraní, které bylo vytvořeno pro~testování výsledků této aplikace.

% \chapter{Zadání}

% Většina zpravodajských serverů často opatřuje publikované články tzv. ilustračními snímky, jejichž úkolem je vizuálně dokreslovat obsah článku a upoutat na něj čtenářovu pozornost. Ilustrační snímky většinou pocházejí z~rozsáhlých fotografických databází, jsou vybírány autory článku a s obsahem článku souvisejí jen relativně volně. Výběr ilustračních snímků probíhá nejčastěji na základě porovnávání klíčových slov specifikovaných autorem textu a popisků, kterými jsou obrázky v~databázi opatřeny (typicky svými autory). 

% Proces výběru ilustračních snímků (dotazování ve~fotografické databázi) je obtížný jednak pro~samotný vyhledávací systém (hledání relevantních fotografií na základě uživatelských dotazů), jednak pro~autory, kteří musí dotazy vytvářet. Konstrukce dotazů spočívá v~několika krocích: uživatel nejdříve musí identifikovat ústřední téma (či témata) článku, které chce ilustrovat vhodnou fotografií, a ta potom popsat vhodnými klíčovými slovy, zvolit a zkombinovat je tak, aby vedla k~nalezení vhodného obrázku. Tento proces by mohl být zjednodušen tím, že konstrukce dotazů pro~vyhledávání bude prováděna automaticky pouze na základě textu článku. 

% Cílem diplomové práce je navržení a implementace komfortní webové aplikace pro~automatické navrhování ilustračních snímků na základě textu článku, bez nutnosti explicitně konstruovat vyhledávací dotazy. Součástí práce bude i uživatelská evaluace celého systému. pro~experimenty bude použita kolekce ilustračních snímků od společnosti Profimedia.

