\chapter{Vyhledávání podobných obrázků}

Jednou ze služeb, které výsledná aplikace poskytuje, je vyhledávání podobných obrázků. Uživatel rozhraní najde pomocí textových dotazů nějaké ilustrační obrázky a má možnost u každého z nalezených obrázků získat obrázky vizuálně podobné. Tato kapitola pojednává o tvorbě backendové služby, která vyhledávání podobných obrázků umožňuje.

Vstupními daty je soubor s vektory pro každý obrázek datasetu Profimedie. Vektor má 4096 složek s reálnými nezápornými čísly. Vektory jsou vizuální deskriptory obrázků. Tyto deskriptory byly vygenerovány pomocí software Caffe a jsou jsou odezvami předposlední vrstvy hluboké konvoluční neuronové sítě natrénované pro klasifikaci obrázků do 1000 kategorii.

Mějme obrázek $I_1$ s deskriptorem $D_1$ a obrázek $I_2$ s deskriptorem $D_2$. Míru podobnosti obrázků $Similarity$ pak můžeme definovat jako

\begin{equation}
  Similarity(I_1, I_2) = \sum_{i=1}^{4096} |D_1[i]-D_2[i]|
\end{equation}

V praxi se dají výsledky této míry klasifikovat zhruba do 3 kategorií. Tyto vypozorované kategorie popisuje tabulka \ref{fig:simtypes}.

\begin{figure}
\label{fig:simtypes}
\centering
\begin{tabular}{ | c | r |}
  \hline
     Kategorie & $Similarity$\\
  \hline
  \hline
    téměř shodné & $0 - 500$ \\
  \hline
    podobné & $500 - 1500$ \\
  \hline
    nepodobné & $> 1500$ \\
\hline
\end{tabular}

  \caption{Kategorie podobnosti obrázků podle $Similarity$.}
\end{figure}

