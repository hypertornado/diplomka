\chapter{Instalace a zprovoznění}
\label{chap:implementace}

Backend aplikace je webová aplikace napsaná v jazyce Ruby a frameworku Ruby on Rails. Je k dispozici pod svobodnou licencí MIT. K jejímu spuštění potřebujete ruby verze alepoň 2.0 (nižší verze nejsou otestované), Javu a ke stažení zdrojového kódu git. Aplikaci jde spustit na platformách Linux a Mac.

\section{Instalace}

Zdrojový kód je volně dostupný na webu GitHubu\footnote{\url{https://github.com/hypertornado/diplomka}}. Stáhnou tedy lze příkazem

\begin{lstlisting}[language=bash]
git clone https://github.com/hypertornado/diplomka
\end{lstlisting}

Tento příkaz vytvoří adresář diplomka. Závislosti aplikace nainstalujete pomocí bundleru:

\begin{lstlisting}[language=bash]
bundle install
\end{lstlisting}

Instalace může vyžadovat přístup administrátora. Dále je potřeba stáhnout knihovnu Elasticsearch\footnote{\url{http://www.elasticsearch.org/downloads/1-0-3/}} do adresáře bin/elasticsearch. Stačí verze 1.0 a vyšší.

Nyní je možné celý projekt spustit pomocí Bash příkazů. Nejprve se spustí elasticsearch databáze pomocí příkazu \lstinline{rake es:start}, poté je možné spustit samotnou aplikaci příkazem \lstinline{rails server}. Po spuštění severu je uživatelské rozhraní dostupné ve webovém prohlížeči na adrese \lstinline{http://localhost:3000}. Po načtení stránky se zobrazí uživatelské rozhraní, ale veškeré AJAXové dotazy skončí chybou. V databázi nejsou importována data.

\section{Práce s metadaty k obrázkům}

Metadata k obrázkům a obrázky samotné jsou poskytovány firmou Profimedia a nejsou volně dostupné. Ke zprovoznění aplikace je nutné vložit CSV soubor \lstinline{keyword-cleaned-phrase-export.csv} do adresáře data.

\section{Překlad metadat}
Soubor obsahuje metadata k obrázkům v angličtině. Jedním z úkolů této práce je poskytnout doporučování obrázků i v jiných jazycích, primárně v českém jazyce. Bylo tedy nutné metadata přeložit. Pokoušeli jsme se použít volný nástroj na překlad Moses. Ve verzi 2.1\footnote{\url{http://www.statmt.org/moses/RELEASE-2.1/models/en-cs/model/}} nabízí volně dostupné modely pro překlad z češtiny do angličtiny. I na SSD disku trvá několik hodin, než se překladový model načte do paměti. Překlad jednoho segmentu s tímto modelem byl poměrně pomalý (překlad metadat k jednomu obrázku trval zhruba 3 sekundy) a také dosti nepřesný. Například řádek

\begin{lstlisting}
"0000000003","little baby smiling","","child children baby babies infants kids childhood single faces body naked nake facial expressions smile smiling viewing watching laying fun amusing amusement amused amuse dallying frolicing playing wantoning open"^M
\end{lstlisting}

byl do čestiny přeložen takto:

\begin{lstlisting}
"0000000003","little|UNK|UNK|UNK dítě smiling","","child|UNK|UNK|UNK dětí , dětské děti kojence děti dětství jednotného čelí orgán nahé nake|UNK|UNK|UNK pořídili vyjádření usmívat usmívá odůvodněním , která zábavné sledovat zábavné zábavných i pobavena tím amuse|UNK|UNK|UNK dallying|UNK|UNK|UNK frolicing|UNK|UNK|UNK hrát wantoning|UNK|UNK|UNK open"^M|UNK|UNK|UNK
\end{lstlisting}

Je vidět poměrně velké množství nepřeložených slov (koncovky \lstinline{|UNK}) a překlad je relativně nepřesný. Je pravděpodobné, že by s lepšími daty šel natrénovat lepší překladový a jazykový model. Strojový překlad není hlavním tématem této práce, takže bylo jednodušší komerční automatický překlad Google Translate, který přeloží ukázková metadata takto:

\begin{lstlisting}
"0000000003", "malé dítě s úsměvem", "", "dítě děti dítě děti kojenci děti dětství jednotlivé plochy těla nahá nake výrazy obličeje, úsměvu, usměvavý sledování sledování kterým zábava zábavné zábavní pobavený pobavit laškoval frolicing hrát wantoning otevřený" ^ M
\end{lstlisting}

I z této ukázky je zřejmé, že Google poskytuje kvalitnější překlad, než anglicko-český překladový model v releasu Mosese. Google poskytuje překlad zdarma přes webové rozhraní. Pokud se ale do překladového formuláře nahraje nespecifikované větší množství dat, přestane překlad fungovat. Google pro překlad poskytuje placené API. Platí se XX amerických dollarů za přeložené slovo. Korpus Profimedie obsahuje zhruba (wc vystup 20119222 347129204 4811998848), takze by cely překlad stál XX dollaru. Jelikož se slova v textu opakuji, lze použít překlad slovo po slovu. Ještě lepší by bylo překládat přímo klíčové fráze. Bohužel korpus Profimedie jednotlivé fráze neodděluje. Je ale možné použít učící algoritmus a klíčové fráze detekovat.

\subsection{Export slov a frází}
Nejprve příkazem \lstinline{rake data:export_profimedia_words_for_translation} vyexportujeme do souboru \lstinline{data/word_list.txt} seznam všech slov použitých v metadatech k obrázkům. Tento příkaz běží několik hodin i na moderním počítači s SSD diskem. Z Profimedia dat získáme seznam 352862 slov. Tento soubor je nutné přeložit z angličtiny do dalších podporovaných jazyků, v našem případě češtiny.

\section{Jazykový korpus}
Tato práce potřebuje jazykové korpusy pro podporované jazyky ze dvou důvodů. První je potřeba pro určení relativní jazykové frekvence slov v algoritmu TF-IDF. Zadruhé je potřeba jazykový korpus pro rozpoznávání jazyků. Přirozeným jazykovým korpusem by mohla být samotná Profimedia data, ale pro oba účely jsou tato data nepoužitelná. Klíčová slova a názvy obrázků jsou odlišnými druhy textů, než je průměrný článek. Je tedy potřeba jazyková data získat jinde.

Wikipedia se jako korpus velmi hodí. Textový obsah je pod licencí Creative Commons a jeho struktura je velmi podobná běžnému publicistickému článku. Data se dají získat stažením přímo ze serverů wikipedie (pro češtinu zde), nebo pomocí sdílených torrentů. Práce wiki dumpy všech podporovaných jazyků v adresáři data pod názvem typu \lstinline{wiki_dump_jazyk.xml}. Pro práci s daty z wikipedie je potřeba nejprve převést XML data do textového formátu. K tomu slouží python skript \lstinline{lib/WikiExtractor.py} od Wikipedie. Pro převod anglických dat lze použít příkaz:

\begin{lstlisting}
rake wiki:extract_words_from_wiki en
\end{lstlisting}

Stejný příkaz je potřeba spustit i pro ostatní podporované jazyky. Příkaz vytvorí v \lstinline{data/wiki_en} adresářovou strukturu. Není potřeba převést všechna data, pouze tolik, abychom dostali reprezentativní korpus. Pro angličtinu stačí převést zhruba 10000 článků, pro podobné množství českých dat je potřeba převést zhruba 20000 článků (údaj o exportovaných článcích je průběžně vypisován na konzoli).

Nyní je možné vytvořit seznam slov v korpusu s frekvencemi. Ve skutečnosti nás zajímají pouze stemy slov, ne jednotlivé tvary. Příkaz

\begin{lstlisting}
rake wiki:frequency_list_from_wiki
\end{lstlisting}

vytvoří seznam slov a jejich frekvencí s cestou \lstinline{data/wiki_freq_list_count.en}. Ve skutečnosti nás nezajímají všechna slova z korpusu wikipedie, ale pouze ta, která se vyskytují mezi klíčovými slovy v Profimedia korpusu. Příkazem

\begin{lstlisting}
rake data:create_tf_df_list
\end{lstlisting}

se z korpusu profimedia dat exportují stemy všech slov v profimedia korpusu spolu z hodnotami udávajícími frekvenci termínu v celém korpusu (TF) a frekvenci toho, v kolika popiskách obrázků se slovo objevilo (DF). Pro angličtinu jsou tato data uložena do souboru \lstinline{data/tf_df_list_en.txt}. Nyní můžeme spárovat informaci o stemech z wikipedie a nahrát je do databáze. Párování se provede příkazem:

\begin{lstlisting}
rake data:pair_profimedia_and_wiki_data
\end{lstlisting}

který vytvoří soubor \lstinline{data/paired_wiki_and_profimedia.txt} se statistikami všech stemů z Profimedia korpusu.


\section{Příprava dat pro detekci jazyků}

Automatická detekce jazyka probíhá ve frontendové části. Ke svému chodu ale potřebuje data z korpusu, konkrétně seznam nejčastějších trigramů pro každý podporovaný jazyk. Příkaz

\begin{lstlisting}
rake trigrams:extract_most_frequent_trigrams
\end{lstlisting}

vytvoří pro každý jazyk seznam padesáti nejčastějších trigramů. Seznam pro angličtinu je uložen v souboru \lstinline{data/most_frequent_trigrams_en.txt}. Příkaz

\begin{lstlisting}
rake trigrams:trigrams_to_javascript_classes
\end{lstlisting}

pracuje se seznamy nejčastějších trigramů ze kterých vytvoří javascriptovou třídu \lstinline{oo.diplomka.Languages.Data} v notaci Google Closure. Ta je uložena v souboru \lstinline{public/js/js/diplomka/languages/data.js}.

\section{Import dat do databáze}

Veškerá data jsou importována do databáze elasticsearch. Aplikace očekává Elasticsearch připojený na portu 9200. Samotná data se importují příkazem:

\begin{lstlisting}
rake es:import_image_metadata
\end{lstlisting}

Tento vytvoří v elasticsearch index diplomka. Poté nastaví explicitní mapování v celém indexu. Pokud se mapování explicitně nenastaví, elasticsearch se snaží data namapovat podle jednoduchých heuristik. Například text rozdělí na slova, která pak převede na malá písmena a upraví defaultním stemmerem. Aplikace se však o stemming a převedení na malá písmena stará sama, mapování tedy říká, aby nahraný text oddělil pouze pomocí mezer na slova a dále nezpracovával. Vyhledávací dotaz je pak rozdělen také pouze pomocí mezer. Nastavené mapování lze ověřit pomocí API elsaticsearch na adrese \url{http://localhost:9200/diplomka/_mapping/}.

Po vytvoření indexu může začít samotný import dat. Data z každého řádku Profimedia dat je převeden do formátu JSON. Data obsahují položky locator, title 

\section{Import metadat obrázků}




