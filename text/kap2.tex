\chapter{Praktická část: Implementace moderní webové aplikace}

Práce je z velké části implementační. Snažil jsem se tedy vytvořit moderní webovou aplikaci s využitím co nejvíce frontendových novinek v novém standardu HTML5.

\section{Databáze: jak uložit 20M metadat obrázků}

Jak uložit takové množství dat, aby se dalo rychle vyhledávat. Škálovatelnost. Dostupnost knihoven pro práci s databázema. Proč jsem si vybral nakonec ES. SQL vs NoSQL.

\section{Backend a úprava dat: Komunikace s databází, implementace algoritmů}

Proč jsem zkoušel golang a proč jsem nakonec použil Ruby a Ruby on Rails. Základní popis MVC frameworku. Dostupnost knihoven pro získání stemů a prací s databází.

\section{Frontend: AJAXová aplikace na zobrazování obrázků}

Jaké jsou dnešní možnosti vývoje frontendu. Single page aplikace. Možnosti moderních prohlížečů. JavaScriptové knihovny. Proč to nedělám v jQuery, ale používám Google Closure. Google Closure Library, Templates, Compiler.

Návrh rozhraní bez jediného tlačítka. Responzivní webdesign.

\section{Anotační rozhraní}

Jak lze v Ruby on Rails vyrobit jednoduše anotační rozhraní s uživateli a s ukládáním do databáze.

\section{Návod k použití}

Popis prvků. Screenshoty aplikace.

\section{NOTES}

Language data jsem stahnul z:
http://dumps.wikimedia.org/cswiki/latest/

Ceska data jsou z:
/net/seznamdata/profiset/profi-text-cleaned.csv

prekladova data:
moses - http://www.statmt.org/moses/RELEASE-2.1/binaries/macosx-mavericks/bin/

prekladovy model:
http://www.statmt.org/moses/RELEASE-2.1/models/en-cs/
