
\chapter{Evaluace výsledků}

\section{Evaluace výsledků}

Výsledkem práce by mělo být rozhraní pro anotaci obrázků využívající algoritmus na hledání klíčových slov v textu.

Tento algoritmus se dá uživatelsky testovat několika způsoby. Uživatel vidí text a několik (cca 5) vrácených obrázků algoritmem. Uživatel vybere množinu relevantních obrázků. Další možností je mezi 5 vrácených obráyků vložit jeden náhodný. Úkolem anotátora je pak vybrat ten náhodně vybraný. Přesnost algoritmu pak jde měřit pomocí toho, kolikrát se anotátor trefí do špatného obrázku (potřeba zdrojový článek)


\section{Metodika}

Jak budu měřit. Porovnání naivního a nejlepšího algoritmu. Nejlepší algoritmus vybrán pomocí oanotovaných klíčových slov ve Wikinews článcích. Pak anotátoři se budou snažit najít nejméně vhodný obrázek z nabízených hodnot.

Evaluace jenom pro angličtinu, nebo i pro češtinu?

\section{Výsledky}

Grafy, tabulky.

\section{Možná zlepšení}

Jak bychom kmohli mít lepší data a co omezuje použitý algoritmus.