\chapter{Backend}

Backend je jednou z klíčových částí aplikace. Kromě obsluhy uživatele statickými soubory (HTML, CSS, Javascript) je jeho hlavní úlohou poskytnou API pro vyhledávání.

\section{Databáze}

Úkolem databáze je uložit data a umožnit jejich prohledávání. Uživatelé této aplikace nemají možnost databázi modifikovat. Zápis do databáze provede administrátor pouze jednou před startem aplikace. Důležitým požadavkem je důraz na rychlost a snadnou škálovatelnost. V posledních několika letech vzniklo mnoho nových databází v kategorii vágně označené jako NoSQL. Tato kategorie databází se těžko popisuje, na každou popsanou vlastnost existuje NoSQL databáze, která tuto podmínku nesplňuje. Obecně ale lze říct že NoSQL databáze nepracují s prvky v tabulkovém uspořádání. Jejich výhodou oproti standardním relačním databázím může být vyšší výkon a snadná škálovatelnost.

Nevýhodou je většinou obtížnější práce s daty. Většina NoSQL databází například mapodporuje databázové transakce a vůbec celý ACID. Pro práci s databázovými daty se často používá model Map Reduce. Algoritmus Map Reduce vyvinula a publikovala společnost Google, která na něj má i patent. Google však oznámil, že algoritmus MapReduce postupně přestává používat\footnote{\url{http://www.datacenterknowledge.com/archives/2014/06/25/google-dumps-mapreduce-favor-new-hyper-scale-analytics-system/}}.

První verze aplikace byla postavena na databázi CouchBase. Výhodou CouchBase je dobrý výkon, velmi dobrá dokumentace a také podpora knihoven pro mnoho jazyků přímo od společnosti Couchbase. Brzy se však ukázalo, že implementace textového vyhledávání v Couchbase by byla velmi náročná.

Vhodnější pro daný účel se ukázala knihovna Elasticsearch. Nejedná se v pravém smyslu o databázi. Jejím hlavním cílem je poskytnout snadné vyhledávání nad daty. Je postavená nad knihovnou Apache Lucene a poskytuje snadnou škálovatelnost. Komunikace mezi knihovnou a klientem probíhá pomocí RESTového HTTP API. Hlavním podporovaným formátem dat je JSON. Elasticsearch poskytuje programátorovi velké možnosti v nastavení prohledávání. V textovém vyhledávání může uživatel databáze použít všechny tokenizery a stemmery z knihovny Lucene. Důležitou vlastností je, že u hledaných slov může uživatel databáze určit váhu jednotlivých slov. V průběhu implementace aplikace navíc vyšla verze knihovny 1.0.

\section{Programovací jazyk}

Se vzrůstající popularitou webů vzniká stále větší množství webových frameworků a dokonce programovacích jazyků zaměřených primárně na programování pro web.

\subsection{NodeJS}
Jedním z nových trendů je tvorba webového backendu v Javascriptu pomocí knihovny NodeJS. Frontendoví vývojáři jsou prakticky nucení Javascript používat. Pokud se v Javascriptu tvoří i backendová část aplikace, může snadněji dojít ke sdílení kódu i pracovních pozic. Trend psaní všech aplikací v Javascriptu glosoval již před sedmi lety Jeff Atwood ve svém pravidle:

\begin{lstlisting}
Any application that can be written in JavaScript, will eventually be written in JavaScript.
Vše co může být napsáno v Javascriptu, bude v Javascriptu napsáno.
\end{lstlisting}

Nevýhody použití Javascriptu jako backendového programovacího jazyka jsou poměrně zřejmé. Javascript byl navržen pro programování webového frontendu, jeho standartní knihovna je v porovnání s ostatními jazyky velmi chudá, podpora objektového programování je celkem nepřímočará. Obsáhlý kód v Javascriptu může být poměrně nepřehledný a jazyk svádí k vytvoření takzvaného \uv{callback hell}, který může mít strukturu jako na obrázku (source http://strongloop.com/strongblog/node-js-callback-hell-promises-generators/). Pokud chce navíc programátor sdílet kód z backendu i na frontendu, musí být kód kompatibilní s podporovanými prohlížeči. Funkce pro jednodušší práci s polem podporuje webový prohlížeč Internet Exporer až od verze 9 [http://kangax.github.io/compat-table/es5/]. V součtu nám převažily nevýhody NodeJS frameworku na výhodami a myšlenku vývoje backendu v Javascriptu jsme opustily.

\begin{lstlisting}
doAsync1(function () {
  doAsync2(function () {
    doAsync3(function () {
      doAsync4(function () {
    })
  })
})
\end{lstlisting}

\subsection{Go}

Go (známý také jako golang) je programovací jazyk od společnosti Google. První stabilní verze byla zveřejněna v Květnu 2012. Go je kompilovaný, staticky typovaný jazyk s garbage collectorem. Syntaxe je inspirována jazykem C a přizpůsobena pro rychlou kompilaci. Velkou výhodou je snadná práce s vlákny pomocí \uv{go rutin}. Programátoři v Javě, nebo C++ může překvapit poněkud netypická podpora práce s objekty.

První verze aplikace byla napsána právě v jazyce Go. V praxi se ukázaly všechny výše uvedené výhody. Hlavní nevýhodou se však ukázal nedostatek kvalitních knihoven. Přestože je jazyk velmi mladý, stal se velmi populárním, o čemž svědčí například počet repozitářů na GitHubu, nebo otázek na StackOverflow. Bude to ovšem trvat ještě nějaký čas, než se knihovny pro go a jejich vývoj stabilizují. Jeden z nyní nejpopulárnějších webových frameworků pro go --- Martini --- v době začátku práce na aplikaci ani neexistoval. Právě nedostatek knihoven pro go vedl k volbě jiného jazyka. Rozhodujícím pro opuštění backendu v jazyce go byla neexistující knihovna pro práci s Elasticsearch. API je sice postaveno na protokolu HTTP, takže šlo s Elasticsearch komunikovat bez použití specializované knihovny, v praxi se to ovšem ukázalo být problematické. Vývoj aplikace si žádal testování různých nastavení a rychlé prototypování v kódu. 

Z interních zdrojů máme informaci o tom, že se Elasticsearch chystá vytvořit knihovnu i pro jazyk Go. Pro naše účely se ukázalo výhodnější přepsat aplikaci do dynamického jazyka s lepší podporou pro Elasticsearch.

\subsection{Ruby a Ruby on Rails}

Ruby je dynamicky typovaný jazyk, silně inspirovaný Perlem s důslednou podporou objektového programování. Popularita Ruby vzrostla zejména kvůli webovému frameworku Ruby on Rails, který je v Ruby napsaný. RoR zpopularizovaly koncept Model-View-Controller při tvorbě webových aplikací. Backend aplikace byl nakonec kompletně přepsán jako aplikace v Ruby on Rails. Elasticsearch poskytuje pro práci s Ruby vlastní knihovnu. Další výhodou se ukázala podpora knihovny Rake, což je jakási obdoba Makefile skriptů v Ruby. Pomocí Rakefilu jdou napsat přehledné úlohy pro manipulaci s daty. 

Standartní implementace jazyka Ruby byla v minulosti kritizována pro svou pomalost. V průběhu práce na této aplikaci vyšla verze Ruby 2.0, která Ruby dosti zrychluje. Tato rychlost byla pro aplikaci shledána dostatečnou.

\section{Komunikace frontend-backend}

Backend poskytuje služby frontendu pomocí API. Existuje několik přístupů a technologií, jak data mezi backendem a fronendem posílat.

\subsection{Formát dat}

V součastnosti jsou pro posílání dat nejběžnější dva formáty -- XML a JSON. XML je klasický formát se spoustou možností a nástrojů. Moderní aplikace však stále více přechází k formátu JSON. Formát JSON je velmi úsporný datový formát, který vychází z datových typů v JavaScriptu. Právě úspornost je jednou z jeho největších výhod oproti XML -- stejné informace mají v JSON typicky kratší zápis než obdoba v XML. Většina moderních browserů umí formát JSON parsovat a práce v JavaScriptu je pak vzhledem ke kompatibilitě datových typů velmi pohodlná.

Zejména kvůli poslednímu důvodu používá tato práce pro komunikaci mezi fronendem a backendem právě formát JSON.

\subsection{REST API}

REST je zkratka pro Representational state transfer. Jedná se o obecnou architekturu rozhraní. V tomto kontextu nás ale zajímá hlavně její navázání na protokol HTTP. REST využívá metody protokolu HTTP pro změnu, nebo získání stavu datových objektů. Ukázkou REST API s použitím formátu JSON může být například knihovna Elasticsearch. Mějme například frekvenční data pro český stem \uv{lid} a instanci Elasticsearch na adrese \lstinline{http://localhost:9200/}. Ukázky HTTP requestů XX - XX ukazují, jak taková data uložit, získat, změnit, nebo smazat.

REST API v kombinaci s JSON fromátem používá tato aplikace jak ke komunikaci mezi frontendem a backendem, tak mezi backendem a databází Elasticsearch.

\subsection{Websocket}

Alternativou k REST API a protokolu HTTP je protokol WebSocket. WebSocket je stejně jako protokol HTTP postaven nad protokolem TCP. Na rozdíl od HTTP ale poskytuje duplexní spojení. Klient je stále spojený se serverem a oba mohou posílat zprávy bez ohledu na druhou stranu. Spojení přes WebSocket má většinou menší latenci než použití protokolu HTTP \footnote{\url{http://www.websocket.org/quantum.html}}. Protokol je podporován všemi moderními verzemi webových prohlížečů a podpora existuje i v knihovnách pro backendové programovací jazyky.

První verze aplikace používali pro komunikaci mezi klientem a serverem právě protokol WebSocket. Nakonec však převážily nevýhody takovéhoto řešení nad výhodami. Jednou z nevýhod je nutnost udržovat spojení s klientem na serverové i klientské straně. Přináší to několik netriviálních problémů. Například v okamžik, kdy se toto spojení přeruší. Pokud se naproti tomu přeruší spojení server-klient při HTTP requestu, může klient zkusit vyslat stejný požadavek znovu.

Druhým problémem je emulace HTTP požadavků v protokolu WebSocket. Z klienta můžeme odeslat HTTP dotaz na server a dostaneme k němu přiřazenou odpověď. V protokolu WebSocket pošleme serveru zprávu a za nějaký čas můžeme dostat zprávu od serveru jako odpověď. Párování došlých zpráv z odeslanými zprávami-požadavky ale musíme implementovat vlastnoručně, například pomocí unikátních ID v těle zprávy. Navíc musíme umět řešit situaci, kdy žádná odpověď ze serveru nedojde. Například nastavením timeoutu pro čekání na odpověď.

WebSocket využijí zejména aplikace, které potřebují, aby server mohl posílat klientovi kdykoliv zprávy, nebo co nejnižší latenci. Takovými aplikacemi mohou být například různé chatovací služby, nebo online hry. Pro jiné většinou asi převáží nevýhody WebSockets oproti HTTP protokolu.

\section{Shrnutí}

Architektura je shrnuta na diagramu XX. Aplikace má backend napsaný v jazyce Ruby a frameworku Ruby on Rails. Data jsou uložená v databázi Elasticsearch. Backend komunikuje s frontendem i databází pomocí REST API, data jsou přenášena ve formátu JSON.








