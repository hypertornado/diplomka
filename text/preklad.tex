\chapter{Tvorba multijazyčného vyhledávání}

Jedním z požadavků na aplikaci je poskytnout doporučení obrázků k textu i pro texty v českém jazyce. Dodaná metadata jsou v jazyce anglickém. Vzhledem k tomu, že metadat k obrázkům je více než 20 milionů, nepřipadá lidský překlad v úvahu z časových i finančních důvodů. Jedinou reálnou možností je použít překlad strojový.

\section{Strojový překlad}

Strojový překlad jako obor zažívá velký rozvoj. Potřeba překladu stále celosvětově prudce stoupá. S tím se zvyšuje poptávka po lepším strojovém překladu a zároveň také vzniká více lidských překladů, které pak jde použít jako zdrojová data k překladům strojovým. Existují různé metody strojového překladu. Na jednu stranu existují pravidlové systémy, které podle pravidel zapsaných překladately převádí text ze zdrojového do cílového jazyka. Na druhou stranu jedním z nejobecnějších řešení je frázový překlad.

\section{Frázový strojový překlad}

Frázový překlad používá oproti pravidlovému přístupu více automatizovaný inženýrský způsob. Ke své práci potřebuje databázi přeložených frází. Fráze jsou několikaslovné kusy přeloženého textu. Typicky se získávají z paralelního korpusu textů ve zdrojovém a cílovém jazyce. Pro extrakci frází z paralelního korpusu existují knihovny jako například GIZA.

Dalším důležitým vstupem strojového frázového systému je korpus cílového jazyka, ze kterého se vytvoří jazykový model. Ten slouží zejména k tomu, aby k sobě poskládané fráze v cílovém jazyce dobře seděli. Pokud máme překladový i jazykový model, můžeme vyjádřit pravděpodobnost překladu pomocí vzorce:

Toto je pouze zjednodušený pohled na statistický frázový překlad. Reálné algoritmy ještě například používají model na reordering slov.

Dobrý frázový překlad potřebuje ke svému chodu miliony frází. Vyhledávání nejpravděpodobnějšího překladu v takovém množství dat je velmi náročná úloha. Nejmodernější algoritmy, které umožňují vyhledávat v překladových datech jsou implementována v knihovně Moses, která je dostupná pod volnou licencí včetně překladových modelů.

\section{Charakteristika dat pro překlad}

Pro správné fungování vyhledávání v českém jazyce je potřeba přeložit klíčová slova v Profimedia datech do češtiny. Slova v korpusu jsou oddělena mezerou. Je ale zřejmé, že některá slova vedle sebe k sobě patří --- jsou to fráze --- zatímco některá nikoliv. Naskytují se tedy zhruba tři možnosti, jak přeložit korpus Profimedie do čestiny.


\subsection{Překlad vět}

První možností je přistupovat k souboru klíčových slov u každého obrázku jako k větě a použít frázový strojový překlad --- buď Moses, nebo Překladač Google --- k překladu z angličtiny do češtiny. Tento přístup má několik problémů. Frázový překlad se snaží aplikovat fráze z překladového modelu. V našem souboru klíčových slov ale mohou být vedle sebe slova, která tvoří frázi pouze zdánlivě. Například můžeme mít fotku dítěte, které stojí před automobilem značky Seat se dvěma klíčovými slovy vedle sebe --- \uv{child} a \uv{seat}. Frázový překlad z angličtiny do češtiny pochopí tato dvě slova jako fráze, které do češtiny přeloží frází \uv{dětské sedadlo}, která ovšem neodpovídá popisku obrázku.

Dalším problémem tohoto přístupu je pomalost. Frázový překlad je dosti náročný algoritmus a překlad dvaceti milionů vět můře být dosti obtížný. Překlad pomocí Mosese nás omezuje výpočetní délkou. Pokud bychom k překladu 20 milionů vět použili Překladač Google, jsme zase omezení cenou za přístup k překladovému API.

\subsection{Překlad slov}

Jednodušším přístupem k překladu klíčových slov je přístup slovníkový, teda překlad každého slova zvlášť. Nejprve je potřeba ze souboru klíčových slov u všech obrázků vyextrahovat všechna slova. Ty pak lze přeložit přímo s použitím slovníku, nebo pomocí frázového strojového překlad (ten použije jednoslovné fráze) Mosesem, či Překladačem Google. Výhodou oproti předchozímu přístupu je menší množství dat a tedy i nižší finanční a časová náročnost takového překladu. Takový systém překladu ale nedokáže detekovat fráze a kvalita překladu je  typicky horší. Vezměmě si například anglická slova \uv{weather} a \uv{vane}. Slovníkový překlad nám slova přeloží jako \uv{počasí} a \uv{lopatka}, lepším překladem by ovšem bylo na obě slova pohlížet jako na anglický překlad českého slova \uv{korouhvička}.

\subsection{Překlad frází}

Poslední navrhovanou možností je oba předchozí principy zkombinovat --- nejprve detekovat v souboru klíčových slov fráze a ty pak přeložit. Detekci frází z korpusu Profimedia provedl ve své XX práci již XX. Zkoušel detekovat 2 a 3 gramy v databázi WordNet a Wikipedii. Výsledky této detekce frází lze použít právě ke zlepšení překladu z angličtiny do cizích jazyků. Na slova která nejsou detekována ve frázi se použije slovníková metoda. Na překlad detekovaných frází lze použít přímo statistický strojový překlad, nebo jeho velmi zjednodušenou variantu. Tato jednodušší varianta pouze projde všechny detekované fráze a podívá se, jestli neexistuje přesně stejná fráze i ve frázovém slovníku překladového modelu. Pokud ano, přeloží detekovanou frázi položkou s nejvyšší pravděpodobností udanou v překladovém modelu.

\section{Závěr překladu}

Překlad klíčových slov z korpusu Profimedia není typickou překladovou úlohou --- nepřekládají se celé věty. Přesto je dokonalý výsledek nemožný. I lidští překladatelé s citem pro jazyk by v této překladové úloze dávali rozdílné výsledky. Strojový překlad zdaleka není na takové úrovni, aby dokázal z širšího kontextu vybrat správný překlad. Lidský překladatel může u překladu klíčových slov využít přímo obrázek, ke kterému se klíčová slova vztahují. Může tak snadněji posoudit, jestli má slovo \uv{single} přeložit jako \uv{jednolůžkový}, nebo ve významu \uv{jeden}. Lepší překlad by mohl přinést překladový model natrénovaný na speciálnější množině dat bližší korpusu Profimedie. Vytvořit takový model by ovšem bylo nad rámec této práce.

Navrhovaný mechanismus překladu poskytuje dostatečně dobrý, i když značně nedokonalý, překlad z angličtiny do češtiny a jednoduše jde zevšeobecnit i pro překlad do dalších jazyků pro které máme potřebná překladová data.

